\section{Results and Discussion}
\par
This section described the results obtained as per the objectives of the project. The section is divided into three main sections;
\begin{itemize}
    \item The discharge flow control unit
    \item The discharge handling unit
    \item Interface and control unit
\end{itemize}
\subsection{The Discharge Flow Control unit}
\par
The design and fabrication process resulted in a  model assembly whose main constituent components are as shown in figure [TODO]. The objective of the discharge flow control unit was to design an automated flow control mechanism that could turn the ball valve in steps of less than one degree which was fully met. A stepper motor was used to achieve the above objective. As seen in figure [TODO], the opening and closing of the ball valve is controlled from the user interface by setting the required number of steps the servo motor has to turn(number of times to perform the experiment) from zero with increments of one. The existing ball valve turns a maximum of 90 degrees for the valve to be fully open. However, as earlier stated, the presence of a hinge at the ball valve interfered with the operation of the motor. The motor thus was calibrated with values within tolerable range to overcome this issue. Figure [TODO] below shows the code snippet of the servo motor calibration used to achieve the above functionality. The calibration from 0 to 76 degrees for the valve to be fully open unlike the normal 0 to 90 degrees means that the motor could turn in less than one degree.
\subsection{The discharge handling unit}
\par
The objective of this section was to design a discharge handling mechanism with incorporated time, weight, and temperature measurements. Figure [TODO] below, shows the handling mechanism. An HX711 temperature probe is used to get real-time temperature values which are sent to the interface. Four load cells have been used to get the weight of the collected discharge. The time is attained automatically by setting the required number of steps to perform the experiment. Figure[TODO] below shows the real values from both the HX711 temperature probe and the load cells as recorded and sent to the user interface.
\subsection{Interface and Control Unit}
\par


\subsection{Final Aseembly}
\par
Figure [TODO] below shows the final assembly inclusive of the discharge flow control unit, discharge handling unit, and the interface and control unit. The unit is designed and fabricated as a plug-and-play, in that it is not permanently fixed onto the machine but can be removed to allow for other experiments to be conducted. Figure[TODO] shows the final assembly with a detailed view.
\subsection{Experiments Conducted}
\par
The fluids experiments were conducted using the system. The objective was to determine the coefficient of discharge from the system and compare it with the one obtained from manually conducting the experiment. This was so as to determine whether the system reduced the human error that resulted from the manual operation of the machine during experiments. 
\par
During the experiment, three runs were conducted in total. With the system in place, it first reduced the number of operators from three to two. Secondly, when conducting the experiment, the user is required to set the number of steps [runs] to perform the experiment and the time. The system then auto-calibrates itself to determine the time allocation for each run. Figure [TODO] below shows screen 2 from the user interface which is used for automatically conducting the experiment. Pressing the start button freezes the back button. The results from each run are recorded via the necessary components and the values sent to be recorded under time, temperature, and weight. The system also indicated the number of runs one is conducting the experiment.
Figure [TODO] below shows the results obtained from the manual operation to determine the coefficient of discharge of the venturi.

