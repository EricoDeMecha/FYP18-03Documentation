\addcontentsline{toc}{section}{Abstract}
\section*{Abstract}
\label{sec:}
\par
The synthetic hydro-experimental machine used for fluid mechanics experiments in the fluids lab at JKUAT uses a manual mechanical system for the collection of the discharge during experiments such as the determination of the coefficient of discharge of the Venturi and the orifice. During such experiments, the user is required to turn the main discharge ball valve in steps determined by human intuition, and for every step, they are required to slide a metallic diverter to collect the discharge to a separate tank, and at the same time, to start measuring the temperature of the discharge, and the timer using an analog stopwatch. This synchronism is necessary for precise data in the computation of the fluid flow properties but cannot be achieved by humans. 
\par
The design and fabrication of an automated discharge collection unit intended to reduce human error by approximately 10\% have been outlined in this report. The design was modularised into three units; a discharge flow control unit, a discharge handling unit, and an interface and a control unit. The discharge flow control unit was designed to control the main ball valve in steps of less than $1^0$ using a servo motor and divert the flow in less than 1.5 seconds using a linear actuator. The discharge handling unit was also designed with a tank that can collect up to 25 kg of discharge. The tank was also fitted with automated temperature and weight measurement units. The interface was designed on a touch LCD running on an STM32 microcontroller.
\par
This automation results in a reduction of the gross error by 8.213 \%.